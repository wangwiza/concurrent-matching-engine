% set font
\usepackage[T1]{fontenc}
\usepackage[utf8]{inputenc}
% \usepackage{mathpazo}
\usepackage{libertine}
% \usepackage{helvet}
% \renewcommand{\familydefault}{\sfdefault}

\usepackage{xcolor}
\usepackage{hyperref,theoremref}
\usepackage[parfill]{parskip}
\hypersetup{
    colorlinks=true, %set true if you want colored links
    linktoc=all,     %set to all if you want both sections and subsections linked
    linkcolor=blue,  %choose some color if you want links to stand out
    citecolor=black,
    filecolor=black,
    linkcolor=black,
    urlcolor=black,
    bookmarksnumbered=true,
    bookmarksopen=true
}

\usepackage{graphicx}
\graphicspath{ {images/} }

\usepackage[outputdir=build]{minted}
\usepackage{mdframed}
\definecolor{bg}{rgb}{0.95,0.95,0.95}
\usepackage{enumitem}

\usepackage{amssymb,amsmath,amsthm,amsfonts}
\usepackage{bm}
\usepackage{mathtools}
\usepackage{multicol,multirow}

\usepackage{varwidth}

% for augmented matrix
\makeatletter
\renewcommand*\env@matrix[1][*\c@MaxMatrixCols c]{%
  \hskip -\arraycolsep
  \let\@ifnextchar\new@ifnextchar
  \array{#1}}
\makeatother

% make all sub items use bullets as well
\setlist[itemize]{label=$\bullet$}

\usepackage{listings}

% Setup the listings package
\lstset{
  basicstyle=\small\ttfamily, % The style of the code
  numbers=left,         % Where to put the line numbers
  numbersep=5pt,        % How far the line numbers are from the code
  tabsize=2,            % Sets default tab size to 4 spaces
  extendedchars=true,
  breaklines=true,      % Sets automatic line breaking
  frame=single,         % Adds a frame around the code
  breakatwhitespace=true
}
\usepackage{tikz}
\usetikzlibrary{tikzmark, arrows.meta, positioning}

% numbered list with only numbers for sub items
\newlist{numlist}{enumerate}{10}
\setlist[numlist]{label*=\arabic*.}

